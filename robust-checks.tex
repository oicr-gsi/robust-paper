\documentclass[10pt,letterpaper]{article}
\usepackage[top=0.85in,left=2.75in,footskip=0.75in]{geometry}

% amsmath and amssymb packages, useful for mathematical formulas and symbols
\usepackage{amsmath,amssymb}

% Use adjustwidth environment to exceed column width (see example table in text)
\usepackage{changepage}

% Use Unicode characters when possible
\usepackage[utf8x]{inputenc}

% textcomp package and marvosym package for additional characters
\usepackage{textcomp,marvosym}

% cite package, to clean up citations in the main text. Do not remove.
\usepackage{cite}

% Use nameref to cite supporting information files (see Supporting Information section for more info)
\usepackage{nameref,hyperref}

% line numbers
\usepackage[right]{lineno}

% ligatures disabled
\usepackage{microtype}
\DisableLigatures[f]{encoding = *, family = * }

% color can be used to apply background shading to table cells only
\usepackage[table]{xcolor}

% array package and thick rules for tables
\usepackage{array}

% enumerate package lets us use letters instead of numbers
\usepackage{enumerate}

% create "+" rule type for thick vertical lines
\newcolumntype{+}{!{\vrule width 2pt}}

% create \thickcline for thick horizontal lines of variable length
\newlength\savedwidth
\newcommand\thickcline[1]{%
  \noalign{\global\savedwidth\arrayrulewidth\global\arrayrulewidth 2pt}%
  \cline{#1}%
  \noalign{\vskip\arrayrulewidth}%
  \noalign{\global\arrayrulewidth\savedwidth}%
}

% \thickhline command for thick horizontal lines that span the table
\newcommand\thickhline{\noalign{\global\savedwidth\arrayrulewidth\global\arrayrulewidth 2pt}%
\hline
\noalign{\global\arrayrulewidth\savedwidth}}


% Remove comment for double spacing
%\usepackage{setspace}
%\doublespacing

% Text layout
\raggedright
\setlength{\parindent}{0.5cm}
\textwidth 5.25in
\textheight 8.75in

% Bold the 'Figure #' in the caption and separate it from the title/caption with a period
% Captions will be left justified
\usepackage[aboveskip=1pt,labelfont=bf,labelsep=period,justification=raggedright,singlelinecheck=off]{caption}
\renewcommand{\figurename}{Fig}

% Use the PLoS provided BiBTeX style
\bibliographystyle{plos2015}

% Remove brackets from numbering in List of References
\makeatletter
\renewcommand{\@biblabel}[1]{\quad#1.}
\makeatother

% Leave date blank
\date{}

% Header and Footer with logo
\usepackage{lastpage,fancyhdr,graphicx}
\usepackage{epstopdf}
\pagestyle{myheadings}
\pagestyle{fancy}
\fancyhf{}
\setlength{\headheight}{27.023pt}
\lhead{\includegraphics[width=2.0in]{PLOS-submission.eps}}
\rfoot{\thepage/\pageref{LastPage}}
\renewcommand{\footrule}{\hrule height 2pt \vspace{2mm}}
\fancyheadoffset[L]{2.25in}
\fancyfootoffset[L]{2.25in}
\lfoot{\sf PLOS}

%% Include all macros below
\newcommand{\fixme}[2]{\textsc{\textbf{{#1}: {#2}}}}
\newcommand{\recommend}[1]{\textit{#1}}
\newcommand{\withurl}[2]{{#1}\footnote{\texttt{#2}}}

\usepackage[ampersand]{easylist}

\begin{document}
\vspace*{0.2in}


\begin{flushleft}
{\Large
\textbf\newline{Ten Simple Rules for Making Research Software More Robust: The Checklist}
}
\newline
\\
{Morgan~Taschuk}\textsuperscript{1,\ddag *},
{Greg~Wilson}\textsuperscript{2,\ddag}
\\
\textbf{1} Ontario Institute for Cancer Research / morgan.taschuk@oicr.on.ca
\\
\textbf{2} Software Carpentry Foundation / gvwilson@software-carpentry.org
\\
\bigskip
{\ddag} These authors contributed equally to this work.
\\
* Corresponding author.
\end{flushleft}


\begin{easylist}[checklist]
& Have a README
&& Explain what the software does
&& List required dependencies
&& Provide compile/installation instructions
&& List input and output files
&& State attributions and licensing
& Print usage information from the command line
&& Include:
&&& the syntax for running the program in GNU/POSIX format
&&& a one line description
&&& the most commonly used arguments, a description of each, and the default values
&&&& Where to find more information
&& Print to standard output
&& Exit with an appropriate exit code
& Version your releases
&& Increment your version number every time you release your software to other people
&& Print when supplying \texttt{-\/-version} or \texttt{-v} on the command line
&& Include version number in output
&& Deposit releases in a stable location so they are available in perpetuity
& Reuse software (within reason)
&& Make sure that you really need the auxiliary program
&& Check for dependent software and version early in execution
&& Use native functions for starting other processes
& Use a build utilty and package manager
&& Document \emph{all} dependencies, preferably in a machine-readable form
&& Avoid depending on scripts and tools which are not available as packages
& Do not require root or other special privileges
&& Allow packages to be installed in an arbitrary location
&& Ask another person to try and build your software
& Eliminate hard-coded paths
&& Set the names and locations of input and output files as command-line parameters
&& Do not require users to navigate to a particular directory to do their work
& Allow configuration of all useful parameters from the command line.
&& Choose reasonable defaults where they exist
&& Set no defaults at all when there aren't any reasonable ones
&& Echo all parameters and software versions to standard out or a log file alongside the results
&& Check that all input values are in a reasonable range near startup
& Include a small test set that can be run to ensure the software is actually working.
&& Tests are easy to find and run
&& Test results are easy to interpret
& Produce identical results when given identical inputs
&& For randomized algorithms: allow the user to optionally provide the seed as an input parameter; or
&& Make sure the acceptable tolerance is known and detailed in documentation and in the tests
\end{easylist}

\section*{Bonus points}

\begin{easylist}[checklist]
& Conform to command-line conventions~\cite{Seemann2013}
& Write high quality documentation~\cite{karimzadeh2016}

\end{easylist}

\bibliography{robust-checks}

\end{document}
