\documentclass[10pt,letterpaper]{article}
\usepackage[top=0.85in,left=1.0in,footskip=0.75in]{geometry}

% Use Unicode characters when possible
\usepackage[utf8x]{inputenc}

% Remove comment for double spacing
\usepackage{setspace}
\onehalfspacing

\usepackage[ampersand]{easylist}

\begin{document}


\begin{flushleft}
{\Large
\textbf\newline{Ten Simple Rules for Making Research Software More Robust: The Checklist}
}
\end{flushleft}


\begin{easylist}[checklist]
& Use version control.
&& Put everything into version control as soon as it is created.
&& Use a feature branch workflow.

& Document your code and usage.
&& Write a good README file.
&& Print usage information.

& Make common operations easy to control.
&& Allow the most commonly changed parameters to be configured from the command line.
&& Check that all input values are in a reasonable range at startup.
&& Choose reasonable defaults where they exist.
&& Set no defaults at all when there aren't any reasonable ones.

& Version your releases.
&& Increment your version number every time you release your software to other people.
&& Make the version of your software easily available by supplying \texttt{-\/-version} or \texttt{-v} on the command line.
&& Include the version number in in all of the program's output.
&& Ensure that old released versions continue to be available.

& Reuse software (within reason).
&& Make sure that you really need the auxiliary program.
&& Ensure the appropriate software and version is available.
&& Ensure that reused software is robust.

& Rely on build tools and package managers for installation.
&& Document all dependencies in a machine-readable form.
&& Avoid depending on scripts and tools which are not available as packages.

& Do not require root or other special privileges to install or run.
&& Do not require root privileges to set up or use packages.
&& Allow packages to be installed in an arbitrary location.
&& Ask another person to try and build your software before releasing it.

& Eliminate hard-coded paths.
&& Set the names and locations of input and output files as command-line parameters.
&& Do not require users to navigate to a particular directory to do their work.

& Include a small test set that can be run to ensure the software is actually working.
&& Make the tests easy to find and run.
&& Make the test script's output easy to interpret.

& Produce identical results when given identical inputs.
&& Echo all parameters and software versions to standard out or a log file alongside the results.
&& Produce the same results each time the same version of the program is run with the same inputs.
&& Allow the user to optionally provide the random seed as an input parameter.
&& Make sure acceptable tolerances are known and detailed in documentation and tests.

\end{easylist}

\end{document}
