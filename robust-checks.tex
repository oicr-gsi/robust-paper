\documentclass[10pt,letterpaper]{article}
\usepackage[top=0.85in,left=2.75in,footskip=0.75in]{geometry}

% amsmath and amssymb packages, useful for mathematical formulas and symbols
\usepackage{amsmath,amssymb}

% Use adjustwidth environment to exceed column width (see example table in text)
\usepackage{changepage}

% Use Unicode characters when possible
\usepackage[utf8x]{inputenc}

% textcomp package and marvosym package for additional characters
\usepackage{textcomp,marvosym}

% cite package, to clean up citations in the main text. Do not remove.
\usepackage{cite}

% Use nameref to cite supporting information files (see Supporting Information section for more info)
\usepackage{nameref,hyperref}

% line numbers
\usepackage[right]{lineno}

% ligatures disabled
\usepackage{microtype}
\DisableLigatures[f]{encoding = *, family = * }

% color can be used to apply background shading to table cells only
\usepackage[table]{xcolor}

% array package and thick rules for tables
\usepackage{array}

% enumerate package lets us use letters instead of numbers
\usepackage{enumerate}

% create "+" rule type for thick vertical lines
\newcolumntype{+}{!{\vrule width 2pt}}

% create \thickcline for thick horizontal lines of variable length
\newlength\savedwidth
\newcommand\thickcline[1]{%
  \noalign{\global\savedwidth\arrayrulewidth\global\arrayrulewidth 2pt}%
  \cline{#1}%
  \noalign{\vskip\arrayrulewidth}%
  \noalign{\global\arrayrulewidth\savedwidth}%
}

% \thickhline command for thick horizontal lines that span the table
\newcommand\thickhline{\noalign{\global\savedwidth\arrayrulewidth\global\arrayrulewidth 2pt}%
\hline
\noalign{\global\arrayrulewidth\savedwidth}}


% Remove comment for double spacing
%\usepackage{setspace}
%\doublespacing

% Text layout
\raggedright
\setlength{\parindent}{0.5cm}
\textwidth 5.25in
\textheight 8.75in

% Bold the 'Figure #' in the caption and separate it from the title/caption with a period
% Captions will be left justified
\usepackage[aboveskip=1pt,labelfont=bf,labelsep=period,justification=raggedright,singlelinecheck=off]{caption}
\renewcommand{\figurename}{Fig}

% Use the PLoS provided BiBTeX style
\bibliographystyle{plos2015}

% Remove brackets from numbering in List of References
\makeatletter
\renewcommand{\@biblabel}[1]{\quad#1.}
\makeatother

% Leave date blank
\date{}

% Header and Footer with logo
\usepackage{lastpage,fancyhdr,graphicx}
\usepackage{epstopdf}
\pagestyle{myheadings}
\pagestyle{fancy}
\fancyhf{}
\setlength{\headheight}{27.023pt}
\lhead{\includegraphics[width=2.0in]{PLOS-submission.eps}}
\rfoot{\thepage/\pageref{LastPage}}
\renewcommand{\footrule}{\hrule height 2pt \vspace{2mm}}
\fancyheadoffset[L]{2.25in}
\fancyfootoffset[L]{2.25in}
\lfoot{\sf PLOS}

%% Include all macros below
\newcommand{\fixme}[2]{\textsc{\textbf{{#1}: {#2}}}}
\newcommand{\recommend}[1]{\textit{#1}}
\newcommand{\withurl}[2]{{#1}\footnote{\texttt{#2}}}

\usepackage[ampersand]{easylist}

\begin{document}
\vspace*{0.2in}


\begin{flushleft}
{\Large
\textbf\newline{Ten Simple Rules for Making Research Software More Robust: The Checklist}
}
\newline
\\
{Morgan~Taschuk}\textsuperscript{1,\ddag *},
{Greg~Wilson}\textsuperscript{2,\ddag}
\\
\textbf{1} Ontario Institute for Cancer Research / morgan.taschuk@oicr.on.ca
\\
\textbf{2} Software Carpentry Foundation / gvwilson@software-carpentry.org
\\
\bigskip
{\ddag} These authors contributed equally to this work.
\\
* Corresponding author.
\end{flushleft}


\begin{easylist}[checklist]
& Use version control.

& Document your code and usage
&& Explain what the software does.
&& List required dependencies.
&& Provide compilation/installation instructions.
&& List input and output files.
&& State attributions and licensing.
&& Print usage information when launching from the command line.
&& Desribe the syntax for running the program.
&& Explain the software's primary function.
&& Describe the most commonly used arguments, with a description of each and default values.
&& Tell users where to find more information.
&& Print to standard output.
&& Exit with an appropriate exit code

& Make common operations easy to control.
&& Allow the most commonly changed parameters to be configured from the command line.
&& Check that all input values are in a reasonable range at startup.
&& Choose reasonable defaults where they exist.
&& Set no defaults at all when there aren't any reasonable ones.

& Version your releases.
&& Increment your version number every time you release your software to other people.
&& Print the version number when given the appropriate flag.
&& Include the version number in all of the program's output.
&& Ensure that old released versions continue to be available.

& Reuse software (within reason).
&& Make sure that you really need auxiliary programs.
&& Ensure the appropriate software and version is available.
&& Ensure that reused software is robust.

& Rely on build tools and package managers for installation.
&& Document all dependencies in a machine-readable form.
&& Avoid depending on scripts and tools which are not available as packages.

& Do not require root or other special privileges to install or run.
&& Packages should not require root privileges to set up or use.
&& Allow packages to be installed in an arbitrary location.
&& Ask another person to try and build your software before releasing it.

& Eliminate hard-coded paths.
&& Set the names and locations of input and output files as command-line parameters.
&& Do not require users to navigate to a particular directory to do their work.

& Include a small test set that can be run to ensure the software is actually working.
&& Make the test script easy to find and run.
&& Make the test script's output easy to interpret.

& Produce identical results when given identical inputs.
&& Echo all parameters and software versions to standard out or a log file alongside the results.
&& A particular version of a program should produce the same results every time it is run.
&& Allow the user to optionally provide the random seed as an input parameter.
&& Make sure the acceptable tolerance is known and detailed in documentation and in the tests.

\end{easylist}

\end{document}
