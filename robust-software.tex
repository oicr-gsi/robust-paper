\documentclass[10pt,letterpaper]{article}
\usepackage[top=0.85in,left=2.75in,footskip=0.75in]{geometry}

% amsmath and amssymb packages, useful for mathematical formulas and symbols
\usepackage{amsmath,amssymb}

% Use adjustwidth environment to exceed column width (see example table in text)
\usepackage{changepage}

% Use Unicode characters when possible
\usepackage[utf8x]{inputenc}

% textcomp package and marvosym package for additional characters
\usepackage{textcomp,marvosym}

% cite package, to clean up citations in the main text. Do not remove.
\usepackage{cite}

% Use nameref to cite supporting information files (see Supporting Information section for more info)
\usepackage{nameref,hyperref}

% line numbers
\usepackage[right]{lineno}

% ligatures disabled
\usepackage{microtype}
\DisableLigatures[f]{encoding = *, family = * }

% color can be used to apply background shading to table cells only
\usepackage[table]{xcolor}

% array package and thick rules for tables
\usepackage{array}

% enumerate package lets us use letters instead of numbers
\usepackage{enumerate}

% create "+" rule type for thick vertical lines
\newcolumntype{+}{!{\vrule width 2pt}}

% create \thickcline for thick horizontal lines of variable length
\newlength\savedwidth
\newcommand\thickcline[1]{%
  \noalign{\global\savedwidth\arrayrulewidth\global\arrayrulewidth 2pt}%
  \cline{#1}%
  \noalign{\vskip\arrayrulewidth}%
  \noalign{\global\arrayrulewidth\savedwidth}%
}

% \thickhline command for thick horizontal lines that span the table
\newcommand\thickhline{\noalign{\global\savedwidth\arrayrulewidth\global\arrayrulewidth 2pt}%
\hline
\noalign{\global\arrayrulewidth\savedwidth}}

\usepackage{color}

% Remove comment for double spacing
%\usepackage{setspace}
%\doublespacing

% Text layout
\raggedright
\setlength{\parindent}{0.5cm}
\textwidth 5.25in
\textheight 8.75in

% Bold the 'Figure #' in the caption and separate it from the title/caption with a period
% Captions will be left justified
\usepackage[aboveskip=1pt,labelfont=bf,labelsep=period,justification=raggedright,singlelinecheck=off]{caption}
\renewcommand{\figurename}{Fig}

% Use the PLoS provided BiBTeX style
\bibliographystyle{plos2015}

% Remove brackets from numbering in List of References
\makeatletter
\renewcommand{\@biblabel}[1]{\quad#1.}
\makeatother

% Leave date blank
\date{}

% Header and Footer with logo
\usepackage{lastpage,fancyhdr,graphicx}
\usepackage{epstopdf}
\pagestyle{myheadings}
\pagestyle{fancy}
\fancyhf{}
\setlength{\headheight}{27.023pt}
\lhead{\includegraphics[width=2.0in]{PLOS-submission.eps}}
\rfoot{\thepage/\pageref{LastPage}}
\renewcommand{\footrule}{\hrule height 2pt \vspace{2mm}}
\fancyheadoffset[L]{2.25in}
\fancyfootoffset[L]{2.25in}
\lfoot{\sf PLOS}

%% Include all macros below
\newcommand{\fixme}[2]{\textsc{\textbf{{#1}: {#2}}}}
\newcommand{\recommend}[1]{\textit{#1}}
\newcommand{\withurl}[2]{{#1}\footnote{\texttt{#2}}}
\newcommand{\rulemajor}[1]{\section*{#1}}
\newcommand{\ruleminor}[1]{\textbf{#1}}
\begin{document}
\vspace*{0.2in}

\begin{flushleft}
{\Large
\textbf\newline{Ten Simple Rules for Making Research Software More Robust}
}
\newline
\\
{Morgan~Taschuk}\textsuperscript{1,\ddag *},
{Greg~Wilson}\textsuperscript{2,\ddag}
\\
\textbf{1} Ontario Institute for Cancer Research / morgan.taschuk@oicr.on.ca
\\
\textbf{2} Software Carpentry Foundation / gvwilson@software-carpentry.org
\\
\bigskip
{\ddag} These authors contributed equally to this work.
\\
* Corresponding author.
\end{flushleft}

\section*{Abstract}

Software produced for research, published and otherwise, suffers from
a number of common problems that make it difficult or impossible to
run outside the original institution, or even off the primary
developer's computer.  We present ten simple rules to make such
software robust enough to be run by anyone, anywhere, and thereby
delight your users and collaborators.

\section*{Author Summary}

Many researchers have found out the hard way that there's a world of
difference between ``works for me on my machine'' and ``works for
other people on theirs''.  Many common challenges can be avoided by
following a few simple rules; doing so not only improves
reproducibility, but can accelerate research.

\section*{Introduction}

Scientific software is typically developed and used by a single
person, usually a graduate student or postdoc~\cite{prins2015}.  It
may produce the intended results in their hands, but what happens when
someone else wants to run it? Everyone with a few years of experience
feels a bit nervous when told to use another person's code to analyze
their data: it will often be undocumented, work in unexpected ways (if
it works at all), rely on nonexistent paths or resources, be tuned for
a single dataset, or simply be an older version than was used in
published papers.  The potential new user is then faced with two
unpalatable options: hack the existing code to make it work, or start
over.

Being unable to replicate results is so common that one publication
refers to it as ``a rite of passage''~\cite{baker2016}.  The root
cause of this problem is that most research software is essentially a
prototype, and therefore is not \emph{robust}. The lack of robustness
in published, distributed software leads to duplicated efforts with
little practical benefit, which slows the pace of
research~\cite{prabhu2011,lawlor2015}.  Bioinformatics software
repositories \cite{ison2016,brazas2012} catalogue dozens to hundreds
of tools that perform similar tasks: for example, in 2016 the
Bioinformatics Links Directory included 84 different multiple sequence
aligners, 141 tools to analyze transcript expression, and 182 pathway
and interaction resources.  Some of these tools are legitimate efforts
to improve the state-of-the-art, but often they are difficult to
install and run~\cite{stajich2002,Seemann2013}, and are effectively
abandoned after publication~\cite{nekrutenko2012}.

This problem is not unique to bioinformatics, or even to
computing~\cite{baker2016}.  Best practices in software engineering
specifically aim to increase software robustness. However, most
bioinformaticians learn what they know about software development on
the job or otherwise informally~\cite{prins2015,atwood2015}.  Existing
training programs and initiatives rarely have the time to cover
software engineering in depth, especially since the field is so broad
and developing so rapidly~\cite{atwood2015,lawlor2015}.  In addition,
making software robust is not directly rewarded in science, and
funding is difficult to come by~\cite{prins2015}. Some proposed
solutions to this problem include restructuring educational programs,
hiring dedicated software engineers~\cite{lawlor2015,sanders2008},
partnering with private sector or grassroots
organizations~\cite{prins2015,ison2016}, or using specific technical
tools like containerization or cloud
computing~\cite{afgan2016,howe2012}. Each of these requires time and,
in some cases, institutional change.

The good news is, you don't need to be a professionally-trained
programmer to write robust software.  In fact, some of the best, most
reliable pieces of software in many scientific communities are written
by researchers~\cite{prabhu2011,sanders2008} who have adopted strong
software engineering approaches, have high standards of
reproducibility, use good testing practices, and foster strong user
bases through constantly evolving, clearly documented, useful, and
useable software.  In the bioinformatics community, Bioconductor and
Galaxy follow this path~\cite{gentleman2004,afgan2016}.  Not all
scientific software needs to be robust~\cite{varoquaux2015}, but if
you publish a paper about your software, it should, at minimum,
satisfy these rules.

So what \emph{is} ``robust'' software?  We implied above that it is
software that works for people other than the original author, on
machines other than its creator's.  More specifically, we mean that:

\begin{itemize}

\item
  it can be installed on more than one computer with relative ease,

\item
  it works consistently as advertised, and

\item
  it can be integrated with other tools.

\end{itemize}

Our rules are generic and can be applied to all languages, libraries,
packages, documentation styles, and operating systems for both
closed-source and open-source software.  They are also necessary steps
toward making computational research replicable and reproducible:
after all, if your tools and libraries cannot be run by others, they
cannot be used to verify your results or as a stepping stone for
future work~\cite{brown2013}.

\rulemajor{Rule 1: Use version control.}

Version control is essential to sustainable software development
\cite{wilson2014,wilson2016}.  In particular, developers will struggle
to understand what they have actually built, what it actually does,
and what they have actually released without some mechanical way to
keep track of changes.  They should therefore \ruleminor{put
everything created manually into version control as soon as it is
created}, including programs, original field observations, and the
source files for papers.  Files that can be regenerated at need, such
as the binaries for compiled programs or intermediate files generated
during data analysis, should not be versioned; instead, it is often
more sensible to use an archiving system for them, and store the
metadata describing their contents in version control
instead~\cite{noble2009}.

If you are new to version control, it is simplest to treat it as ``a
better Dropbox'' (or, if you are of a
certain age, a better FTP), and to use it simply to synchronize files
between multiple developers and machines~\cite{blischak2016}.  Once
you are comfortable working that way, you should \ruleminor{use a
feature branch workflow}: designate one parallel copy (or
``branch'') of the repository as the master, and create a new branch
from it each time you want to fix a bug or add a new feature.  This
allows work on independent changes to proceed in isolation; once the
work has been completed and tested, it can be merged into the master
branch for release.

\rulemajor{Rule 2: Document your code and usage}

How to write high quality documentation has been described
elsewhere~\cite{karimzadeh2016} and so here we only cover two minimal
types: the README and usage.  The README is usually available even
before the software is installed, exists to get a new user started,
and points them towards more help. Usage is a terse, informative
command-line help message that guides the user in the correct use of
the software.

Numerous guidelines exist on how to \ruleminor{write a good README
file}~\cite{Johnson1997,gnustandards}.  At a minimum, your README
should:

\begin{enumerate}

\item
  \textit{Explain what the software does}.  There's nothing more
  frustrating than downloading and installing something only to find
  out that it doesn't do what you thought it did.

\item
  \textit{List required dependencies}.  We address dependencies in
  more detail in Rule~5.

\item
  \textit{Provide compilation or installation instructions}.

\item
  \textit{List all input and output files}, even those considered
  self-explanatory.  Link to specifications for standard formats and
  list the required fields and acceptable values in other files.  If
  there is no rigorous definition for a format, explain its parts as
  clearly as possible in plain English.

\item
  \textit{List a few example commands} to get a user started quickly.

\item
  \textit{State attributions and licensing}. Attributions are how you
  credit your contributors; licenses dictate how others may use and
  need to credit your work.

\end{enumerate}

The program should also \ruleminor{print usage information} when
launching from the command line.  Usage provides the first line of
help for both new and experienced users.  Terseness is important:
usage that extends for multiple screens is difficult to read or refer
to on the fly.

Almost all command-line applications use a combination of
POSIX~\cite{posix2016} and GNU~\cite{gnustandards} standards for
usage.  More standard command-line behaviours are detailed in
\cite{Seemann2013}.  Your software's usage should:

\begin{enumerate}

\item
  \textit{Describe the syntax for running the program}, including the
  name of the program, the relative location of optional and required
  flags, other arguments, and values for execution.

\item
  \textit{Give a short description} to remind users of the software's
  primary function.

\item
  \textit{List the most commonly used arguments}, a description of
  each, and the default values.

\item
  \textit{State where to find more information}.

\end{enumerate}

Usage should be printed to standard output so that it can be combined
with other bash utilities like \texttt{grep}, and it should finish
with an appropiate exit code.

Documentation beyond the README and usage is up to the developer's
discretion.  We think it is very important for developers to document
their work, but our experience is that people are unlikely do it
during normal development However, it is worth noting that software
that is widely used and contributed to has and enforces the need for
good documentation~\cite{gentleman2004}.

\rulemajor{Rule 3: Make common operations easy to control.}

Being able to change parameters on the fly to determine if and how
they change the results is important as your software gains more
users, as it facilitates exploratory analysis and parameter sweeping.
Programs should therefore \ruleminor{allow the most commonly changed
parameters to be configured from the command line}.

Users will want to change some values more often than others.  Since
parameters are software-specific, the appropriate 'tunable' ones
cannot be detailed here, but a short list includes input and reference
files and directories, output files and directories, filtering
parameters, random number generation seeds, and alternatives such as
compressing results, use a variant algorithm, or verbose output.

\ruleminor{Check that all input values are in a reasonable range at
startup}.  Few things are as annoying as having a program announce
after running for two hours that it isn't going to save its results
because the requested directory doesn't exist.

To make programs even easier to use, \ruleminor{choose reasonable
defaults where they exist} and \ruleminor{set no defaults at all
when there aren't any reasonable ones}.  You can set reasonable
default values as long as any command line arguments override those
values.

Changeable values should \emph{never} be hard-coded: if users have to
edit your software in order to run it, you have done something wrong.
Changeable but infrequently-changed values should therefore be stored
in configuration files.  These can be in a standard location,
e.g. \texttt{.packagerc} in the user's home directory, or provided on
the command line as an additional argument.  Configuration files are
often created during installation to set up such things as server
names, network drives, and other defaults for your lab or institution.

\rulemajor{Rule 4: Version your releases.}

Software evolves over time, with developers adding or removing
features as need dictates. Making official releases stamps a
particular set of features with a project-specific identifier so that
version can be retrieved for later use. For example, if a paper is
published, the software should be released at the same time so that
the results can be reproduced.

Most software has a version number composed of a decimal number that
increments as new versions are released.  There are many different
ways to construct and interpret this number, but most importantly for
us, a particular software version run with the same parameters should
give identical results no matter when it's run. Results include both
correct output as well as any errors.  \ruleminor{Increment your
version number every time you release your software to other
people}.

Semantic versioning~\cite{semver} is one of the most
common types of versioning for open-source software. Version numbers
take the form of \emph{MAJOR.MINOR{[}.PATCH{]}}, e.g., 0.2.6.  Changes
in the major version number herald significant changes in the software
that are not backwards compatible, such as changing or removing
features or altering the primary functions of the software. Increasing
the minor version represents incremental improvements in the software,
like adding new features. Following the minor version number can be an
arbitrary number of project-specific identifiers, including patches,
builds and qualifiers.  Common qualifiers include \texttt{alpha},
\texttt{beta}, and \texttt{SNAPSHOT}, for applications that are not
yet stable or released, and \texttt{-RC} for release candidates prior
to an official release.

\ruleminor{The version of your software should be easily available by
supplying \texttt{-\/-version} or \texttt{-v} on the command
line}. This command should print the software name and version
number, and it should also be \ruleminor{included in all of the
program's output}, particularly debugging traces.  If someone needs
help, it's important that they be able to tell whoever's helping them
which version of the software they're using.

While new releases may make a program better in general, they can
simultaneously create work for someone who integrated the old version
into their own workflow a year or two ago, and won't see any benefits
from upgrading.  A program's authors should therefore
\ruleminor{ensure that old released versions continue to be
available.}  A number of mechanisms exist for controlled release
that range from as simple as adding an appropriate commit message or
tag to version control~\cite{blischak2016}, to official releases
alongside code on Bitbucket or GitHub, to depositing into a repository
like apt, yum, homebrew, CPAN, etc. Choose the method that best suits
the number and expertise of users you anticipate.

\rulemajor{Rule 5: Reuse software (within reason)}

In the spirit of code reuse and interoperability, developers often
want to reuse software written by others.  With a few lines, a call is
made out to another library or program and the results are
incorporated into the primary script. Using popular projects reduces
the amount of code that needs to be maintained and leverages the work
done by the other software.

Unfortunately, reusing software (whether software libraries or
separate executables) introduces dependencies, which can bring their
own special pain. The interface between two software packages can be a
source of considerable frustration: all too often, support requests
descend into debugging errors produced by the other project due to
incompatible libraries, versions, or operating
systems~\cite{brown2013}. Even introducing libraries in the same
programming language can rely on software installed in the
environment, and the problem becomes much more difficult when relying
on executables, or even on web services.

Despite these problems, software developers in research should re-use
existing software provided a few guidelines are adhered to.

First, \ruleminor{make sure that you really need the auxiliary
program}. If you are executing GNU sort instead of figuring out how
to sort lists in Python, it may not be worth the pain of
integration. Reuse software that offers some measurable improvement to
your project.

Second, if launching an executable, \ruleminor{ensure the appropriate
software and version is available}.  Either allow the user to
configure the exact path to the package, distribute the program with
the dependent software, or download it during installation using your
package manager. If the executable requires internet access, check for
that early in execution.

Third, \ruleminor{ensure that reused software is robust}. Relying on
erratic third party libraries or software is a recipe for
tears. Prefer software that follows good software development
practices, is open for support questions, and is available from a
stable location or repository using your package manager.

Exercise caution especially when transitioning across languages or
using separate executables, as they tend to be especially sensitive to
operating systems, environments, and locales.

\rulemajor{Rule 6: Rely on build tools and package managers for installation.}

To compile code, deploy applications, and automate other tasks,
programmers routinely use build tools like Make, Rake, Maven, Ant or
MS Build.  These tools can also be used to manage runtime
environments, i.e., to check that the right versions of required
packages are installed and install or upgrade them if they are not.
As mentioned in Rule~5, a package manager can mitigate some of the
difficulties in software reuse.

The same tools can and should be used to manage runtime environments
on users' machines as well.  Accordingly, developers should
\ruleminor{document all dependencies in a machine-readable form}.
Package managers like apt and yum are available on most Unix-like
systems, and application package managers exist for specific languages
like Python (pip), Java (Maven/Gradle), and Ruby (RubyGems). These
package managers can be used together with the build utility to ensure
that dependencies are available at compile/run time.

For example, it is common for Python projects to include a file called
\texttt{requirements.txt} that lists the names of required libraries,
along with version ranges:

\begin{verbatim}
requests>=2.0
pygithub>=1.26,<=1.27
python-social-auth>=0.2.19,<0.3
\end{verbatim}

This file can be read by the pip package manager, which can check that
the required software is available and install it if it is not.
Whatever is used, developers should \emph{always} install dependencies
using their dependency description, especially on their personal
machines, so that they're sure it works.

Conversely, developers should \ruleminor{avoid depending on scripts
and tools which are not available as packages}.  In many cases, a
program's author may not realize that some tool was built locally, and
doesn't exist elsewhere. At present, the only sure way to discover
such unknown dependencies is to install on a system administered by
someone else and see what breaks. As use of virtualization containers
becomes more widespread, software installation can also be tested on a
virtual machine or container system like Docker.

\rulemajor{Rule 7: Do not require root or other special privileges to install or run.}

Root (also known as ``superuser'' or ``admin'') is a special account
on a computer that has (among other things) the power to modify or
delete system files and user accounts. Conversely, files and
directories owned by root usually cannot be modifed by normal users.

Installing or running a program with root privileges is often
convenient, since doing so automatically bypasses all those pesky
safety checks that might otherwise get in the user's way. However,
those checks are there for a reason: scientific software packages may
not intentionally be malware, but one small bug or over-eager
file-matching expression can certainly make them behave as if they
were. Outside of very unusual circumstances, \ruleminor{packages
should not require root privileges to set up or use}.

Another reason for this rule is that users may want to try out a new
package before installing it system-wide on a cluster. Requiring root
privileges will frustrate such efforts, and thereby reduce uptake of
the package. Requiring, as Apache Tomcat does, that software be
installed under its own user account---i.e., that \texttt{packagename}
be made a user, and all of the package's software be installed in that
pseudo-user's space---is similarly limiting, and makes side-by-side
installation of multiple versions of the package more difficult.

Developers should therefore \ruleminor{allow packages to be installed
in an arbitrary location}, e.g., under a user's home directory in
\texttt{\textasciitilde{}/packagename}, or in directories with
standard names like \texttt{bin}, \texttt{lib}, and \texttt{man} under
a chosen directory. If the first option is chosen, the user may need
to modify her search path to include the package's executables and
libraries, but this can (more or less) be automated, and is much less
risky than setting things up as root.

Testing the ability to install software has traditionally been
regarded as difficult, since it necessarily alters the machine on
which the test is conducted.  Lightweight virtualization containers
like Docker make this much easier as well, or simply \ruleminor{ask
another person to try and build your software before releasing it}.

\rulemajor{Rule 8: Eliminate hard-coded paths.}

It's easy to write software that reads input from a file called
\texttt{mydata.csv}, but also very limiting. If a colleague asks you
to process her data, you must either overwrite your data file (which
is risky) or edit your code to read \texttt{otherdata.csv} (which is
also risky, because there's every likelihood you'll forget to change
the filename back, or will change three uses of the filename but not a
fourth).

Hard-coding file paths in a program also makes the software harder to
run in other environments. If your package is installed on a cluster,
for example, the user's data will almost certainly \emph{not} be in
the same directory as the software, and the folder
\texttt{C:\textbackslash{}users\textbackslash{}yourname\textbackslash{}}
will probably not even exist.

For these reasons, users should be able to \ruleminor{set the names
and locations of input and output files as command-line parameters}.
This rule applies to reference data sets as well as the user's own
data: if a user wants to try a new gene identification algorithm using
a different set of genes as a training set, she should not have to
edit the software to do so.  A corollary to this rule is \ruleminor{do
not require users to navigate to a particular directory to do their
work}, since ``where I have to be'' is just another hard-coded path.

In order to save typing, it is often convenient to allow users to
specify an input or output \emph{directory}, and then require that
there be files with particular names in that directory. This practice
is an example of ``convention over configuration'', a principle is
used by software frameworks such as WordPress and Ruby on Rails that
often strikes a good balance between adaptability and consistency.

\rulemajor{Rule 9: Include a small test set that can be run to ensure the software is actually working.}

Every package should come with a set of tests for users to run after
installation. Its purpose is not only to check that the software is
working correctly (although that is extremely helpful), but also to
ensure that it works at all. This test script can also serve as a
working example of how to run the software.

In order to be useful, \ruleminor{make the tests easy to find and
run}.  Many build systems will also run unit tests if provided them
at compile time.  For users, or if the build system is not amenable to
testing, provide a working script in the project's root directory
named \texttt{runtests.sh} or something equally obvious.  This lets
new users build their analysis from a working script.  For example,
with its distribution, HISAT2 includes a full set of very small files,
and a 'Getting Started with HISAT2' section in its manual that leads
you through the entire data lifecycle~\cite{pertea2016}.

Equally, \ruleminor{make the test script's output easy to
interpret}. Screens full of correlation coefficients do not qualify:
instead, the script's output should be simple to understand for
non-experts, such as one line per test, with the test's name and its
pass/fail status, followed by a single summary line saying how many
tests were run and how many passed or failed. If many or all tests
fail because of missing dependencies, that fact should be displayed
once, clearly, rather than once per test, so that users have a clear
idea of what they need to fix and how much work it's likely to take.

Research has shown that the ease with which people can start making
contributions is a strong predictor of whether they will or
not~\cite{steinmacher2015}.  By making it simpler for outsiders to
contribute, a test suite of any kind also makes it more likely that
they will, and software with collaborators stands a better chance of
surviving in the busy field of scientific software.

\rulemajor{Rule 10: Produce identical results when given identical inputs.}

The usage message tells users what the program could do.  It is
equally important for the program to tell users what it actually did.
Accordingly, when the program starts, it should \ruleminor{echo all
parameters and software versions to standard out or a log file
alongside the results} to increase the reproducibility of that step.

Given a set of parameters and a dataset, \ruleminor{a particular
version of a program should produce the same results every time it
is run} to aid testing, debugging, and reproducibility.  Even minor
changes to code can cause minor changes in output because of
floating-point issues, which means that getting exactly the same
output for the same input and parameters probably won't work during
development, but it should still be a goal for people who have
deployed a specific version.

Many applications rely on randomized algorithms to improve performance
or runtimes. As a consequence, results can change between runs, even
when provided with the same data and parameters. By its nature, this
randomness renders strict reproducibility and therefore debugging more
difficult. If even the small test set (\#9) produces different results
for each run, new users may not be able to tell whether the software
is working properly. When comparing results between versions or after
changing parameters, even small differences can confuse or muddy the
comparison. And especially when producing results for publications,
grants or diagnoses, any analysis should be absolutely reproducible.

Given the size of biological data, it is unreasonable to suggest that
random algorithms be removed. However, most programs use a
pseudo-random number generator, which uses a starting seed and an
equation to approximate random numbers. Setting the seed to a
consistent value can remove randomness between runs. \ruleminor{Allow
the user to optionally provide the random seed as an input
parameter}, thus rendering the program deterministic for those cases
where it matters. If the seed is set internally (e.g., using clock
time), echo it to the output for re-use later.  If setting the seed is
not possible, \ruleminor{make sure the acceptable tolerance is known
and detailed in documentation and in the tests}.

\section*{Conclusion}

There has been extended discussion over the past few years of the
sustainability of research software, but this question is meaningless
in isolation: any piece of software can be sustained if its users are
willing to put in enough effort.  The real equation is the ratio
between the skill and effort available, and the ease with which
software can be installed, understood, used, maintained, and extended.
Following the ten rules we outline here reduce the denominator, and
thereby enable researchers to build on each other's work more easily.

That said, not \emph{every} coding effort needs to be engineered to
last.  Code that is used once to answer a specific question related to
a specific dataset doesn't require comprehensive documentation or
flexible configuration, and the only sensible way to test it may well
be to run it on the dataset in question. Exploratory analysis is an
iterative process that is developed quick and revised
often~\cite{lawlor2015,sanders2008}.  However, if a script is dusted
off and run three or four times for slightly different purposes, is
crucial to a publication or a lab, or being passed on to someone else,
it may be time to make your software more robust.

\section*{Supporting information}

\emph{S1 Checklist. Robust software checklist.} A checklist summarizing these ten simple rules to apply to your own
software.

\begin{thebibliography}{10}

\bibitem{prins2015}
Prins P, de~Ligt J, Tarasov A, Jansen RC, Cuppen E, Bourne PE.
\newblock Toward effective software solutions for big biology.
\newblock Nature Biotechnology. 2015;33(7):686--687.
\newblock doi:{10.1038/nbt.3240}.

\bibitem{baker2016}
Baker M.
\newblock 1,500 scientists lift the lid on reproducibility.
\newblock Nature. 2016;533(7604):452--454.
\newblock doi:{10.1038/533452a}.

\bibitem{prabhu2011}
Prabhu P, Jablin TB, Raman A, Zhang Y, Huang J, Kim H, et~al.
\newblock A Survey of the Practice of Computational Science.
\newblock In: State of the Practice Reports. SC '11. New York, NY, USA: ACM;
  2011. p. 19:1--19:12.

\bibitem{lawlor2015}
Lawlor B, Walsh P.
\newblock Engineering bioinformatics: building reliability, performance and
  productivity into bioinformatics software.
\newblock Bioengineered. 2015;6(4).
\newblock doi:{10.1080/21655979.2015.1050162}.

\bibitem{ison2016}
Ison J, Rapacki K, Ménager H, Kalaš M, Rydza E, Chmura P, et~al.
\newblock Tools and data services registry: a community effort to document
  bioinformatics resources.
\newblock Nucleic Acids Research. 2016;44(D1):D38--D47.
\newblock doi:{10.1093/nar/gkv1116}.

\bibitem{brazas2012}
Brazas MD, Yim D, Yeung W, Ouellette BFF.
\newblock A decade of web server updates at the bioinformatics links directory:
  2003-–2012.
\newblock Nucleic Acids Research. 2012;doi:{10.1093/nar/gks632}.

\bibitem{stajich2002}
Stajich JE, Block D, Boulez K, Brenner SE, Chervitz SA, Dagdigian C, et~al.
\newblock The Bioperl Toolkit: Perl Modules for the Life Sciences.
\newblock Genome Research. 2002;12(10):1611--1618.
\newblock doi:{10.1101/gr.361602}.

\bibitem{Seemann2013}
Seemann T.
\newblock Ten recommendations for creating usable bioinformatics command line
  software.
\newblock GigaScience. 2013;2(1):15.
\newblock doi:{10.1186/2047-217X-2-15}.

\bibitem{nekrutenko2012}
Nekrutenko A, Taylor J.
\newblock Next-generation sequencing data interpretation: enhancing
  reproducibility and accessibility.
\newblock Nature ReviewsGenetics. 2012;13(9):667--72.
\newblock doi:{10.1038/nrg3305}.

\bibitem{atwood2015}
Atwood TK, Bongcam-Rudloff E, Brazas ME, Corpas M, Gaudet P, Lewitter F, et~al.
\newblock GOBLET: The Global Organisation for Bioinformatics Learning,
  Education and Training.
\newblock PLoS Computational Biology. 2015;doi:{10.1371/journal.pcbi.1004143}.

\bibitem{sanders2008}
Sanders R, Kelly D.
\newblock Dealing with Risk in Scientific Software Development.
\newblock Software, IEEE. 2008;25(4):21--28.
\newblock doi:{10.1109/ms.2008.84}.

\bibitem{afgan2016}
Afgan E, Baker D, van~den Beek M, Blankenberg D, Bouvier D, Čech M, et~al.
\newblock The Galaxy platform for accessible, reproducible and collaborative
  biomedical analyses: 2016 update.
\newblock Nucleic Acids Research. 2016;44(W1):W3--W10.
\newblock doi:{10.1093/nar/gkw343}.

\bibitem{howe2012}
Howe B.
\newblock Virtual Appliances, Cloud Computing, and Reproducible Research.
\newblock Computing in Science Engineering. 2012;14(4):36--41.
\newblock doi:{10.1109/MCSE.2012.62}.

\bibitem{gentleman2004}
Gentleman RC, Carey VJ, Bates DM, Bolstad B, Dettling M, Dudoit S, et~al.
\newblock Bioconductor: open software development for computational biology and
  bioinformatics.
\newblock Genome Biology. 2004;5(10):1--16.
\newblock doi:{10.1186/gb-2004-5-10-r80}.

\bibitem{varoquaux2015}
Varoquaux G. Software for reproducible science: let’s not have a
  misunderstanding; 2015.
\newblock
  \url{http://gael-varoquaux.info/programming/software-for-reproducible-science-lets-not-have-a-misunderstanding.html}.

\bibitem{brown2013}
Brown CT. Replication, reproduction, and remixing in research software; 2013.
\newblock \url{http://ivory.idyll.org/blog/research-software-reuse.html}.

\bibitem{wilson2014}
Wilson G, Aruliah DA, Brown CT, Hong NPC, Davis M, Guy RT, et~al.
\newblock Best Practices for Scientific Computing.
\newblock PLoS Biology. 2014;12(1):e1001745.
\newblock doi:{10.1371/journal.pbio.1001745}.

\bibitem{wilson2016}
Wilson G, Bryan J, Cranston K, Kitzes J, Nederbragt L, Teal TK.
\newblock Good Enough Practices in Scientific Computing.
\newblock arxivorg. 2016;abs/1609.00037.

\bibitem{noble2009}
Noble WS.
\newblock {A Quick Guide to Organizing Computational Biology Projects}.
\newblock PLoS Computational Biology. 2009;5(7).
\newblock doi:{10.1371/journal.pcbi.1000424}.

\bibitem{blischak2016}
Blischak JD, Davenport ER, Wilson G.
\newblock A Quick Introduction to Version Control with Git and GitHub.
\newblock PLOS Computational Biology. 2016;12(1):1--18.
\newblock doi:{10.1371/journal.pcbi.1004668}.

\bibitem{karimzadeh2016}
Karimzadeh M, Hoffman MM. Creating great documentation for bioinformatics
  software; 2016.
\newblock \url{http://hdl.handle.net/1807/73111}.

\bibitem{Johnson1997}
Johnson M.
\newblock Building a Better ReadMe.
\newblock Technical Communication. 1997;44(1):28--36.

\bibitem{gnustandards}
{\relax Free Software Foundation}. GNU Coding Standards; 2016.
\newblock \url{https://www.gnu.org/prep/standards/standards.html}.

\bibitem{posix2016}
{\relax The IEEE}, {\relax The Open Group}. The Open Group Base Specifications
  Issue 7 IEEE Std 1003.1-2008. 12. Utility Conventions; 2016.
\newblock
  \url{http://pubs.opengroup.org/onlinepubs/9699919799/basedefs/V1_chap12.html}.

\bibitem{pertea2016}
Pertea M, Kim D, Pertea GM, Leek JT, Salzberg SL.
\newblock Transcript-level expression analysis of RNA-seq experiments with
  HISAT, StringTie and Ballgown.
\newblock Nature Protocols. 2016;11(9):1650--1667.

\bibitem{steinmacher2015}
Steinmacher I, Silva MAG, Gerosa MA, Redmiles DF.
\newblock A systematic literature review on the barriers faced by newcomers to
  open source software projects.
\newblock Information and Software Technology. 2015;59:67 -- 85.
\newblock doi:{http://dx.doi.org/10.1016/j.infsof.2014.11.001}.

\end{thebibliography}

\end{document}
